\documentclass{beamer}
\usepackage{listings}
\usepackage{xcolor}
%\usepackage[utf8]{inputenc}
\lstset{language=haskell}
\lstset{moredelim=**[is][\color{blue}]{@}{@}}
\begin{document}
\title{Semantics of mutable state}
\author{Koen Wermer \and Robert Hensing}
\frame{\titlepage}
\frame{\frametitle{What}

\begin{itemize}
 \item A model language with mutable state
 \item Hoare type system
 \item Small step semantics
 \item Type safety proof
 \item Normalization proof
 \item Some neat Hoare logic theorems
\end{itemize}

}

\begin{frame}[fragile]
\frametitle{Basic semantics for mutable state (1)}

\begin{lstlisting}
data Type : Set where
  ...
  <>      : Type
  POINTER : Type -> Type
\end{lstlisting}

\end{frame}

\begin{frame}
\frametitle{Basic semantics for mutable state (2)}

New terms

\begin{itemize}
  \item create cell with value
  \item update cell with value
  \item read cell value
  \item redirect pointer to other cell
  \item ;
  \item the actual cell
\end{itemize}
\end{frame}

\begin{frame}[fragile]
\frametitle{Basic semantics for mutable state (3)}
\begin{lstlisting}
data State : TypeEnv -> Set where
  state : (f : TypeEnv) -> ((p : Pointer)
             -> (Term (f p))) -> State f

data Step  : {ty : Type} @{f : TypeEnv}@
               -> @State f@ -> Term ty
               -> @State f@ -> Term ty -> Set where
     ...
\end{lstlisting}

\end{frame}

\begin{frame}\frametitle{Type system}

Hoare logic

\[ 
   \{ P \}\ c\ \{ Q \}
\]

\only<2>{
\[
\frac{\{ P \}\ c_1\ \{ Q \},\ \ \{ Q \}\ c_2\ \{ R \}  }
     {\{ P \}\ c_1\ ;\ c_2\ \{ R \}  }
\]
}
\end{frame}


\begin{frame}\frametitle{Type system}

`Hoare Type Theory'

\[ 
   \{ P \}\ c:t\ \{ Q \}
\]

\[
\frac{\{ P \}\ c_1:t_1\ \{ Q \},\ \ \{ Q \}\ c_2:t_2\ \{ R \}  }
     {\{ P \}\ (c_1;c_2)\ :\ t_2\ \{ R \}  }
\]
\end{frame}

\begin{frame}\frametitle{Type system}

Separation

``Modular reasoning''

\[
\frac{\{ P \}\ c\ \{ Q \},\ \ modifiedBy(c) \cap freeVars(R) = \emptyset  }
     {\{ P * R \}\ c\ \{ Q * R \}  }
\]
\end{frame}

\begin{frame}\frametitle{Limitations}

\begin{itemize}
 \item Our work: No $\lambda$
 \item `Fundamental': No programs in heap\ \ (or non-termination)
 \item Practical: Proving stuff is hard :)
\end{itemize}

\end{frame}

\begin{frame}\frametitle{Questions?}

Literature
\begin{itemize}
\item Swierstra, Wouter. A functional specification of effects. Diss. University of Nottingham, 2009.
\item Pierce, Benjamin C. Types and programming languages. MIT press, 2002.
\item Nanevski, Aleksandar, Greg Morrisett, and Lars Birkedal. Hoare type theory, polymorphism and separation. Journal of Functional Programming 18.5-6 (2008): 865-911.
\item John C. Reynolds. Separation Logic: A Logic for Shared Mutable Data Structures. LICS 2002.
\end{itemize}

\end{frame}

\end{document}
