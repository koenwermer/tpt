\documentclass{article}
\usepackage{listings}
\usepackage{xcolor}
%\usepackage[utf8]{inputenc}
\lstset{language=haskell}
\lstset{moredelim=**[is][\color{blue}]{@}{@}}
\newcommand{\sectionref}[1]{section \ref{sec:#1}}
\author{Koen Wermer 3705951 \and Robert Hensing 3361063}
\title{Project report TPT}
\begin{document}

\maketitle

\tableofcontents

% TODO read project report instruction from site

% TODO turn these into sections :)
%
%    Research question. Identify a clear and motivated question to answer.
%    Research contribution. Develop an implementation or new knowledge to answer the research question.
%    Research method. Discuss how you arrived at your answer.
%    Related work. Compare your answer with other contributions.

\section{Introduction}

For this project, we set out to explore modeling of the semantics of a
language with mutable state using Agda as a proof checker.

We have extended a simplistic language with only boolean and number
types and no lambda abstractions or any other sort of control flow
structures.
An inductive definition of the language is provided in
\sectionref{def}.
In doing so, and proving theorems about this language, we have
encountered some challenges, which we will describe in
\sectionref{findings}.
Some possibilities we would have liked to explore, but were too
ambitious given the time frame, are described in \sectionref{further}.

\section{Language definition}\label{sec:def}

% TODO

\section{Findings}\label{sec:findings}

\subsection{Small step semantics}

\subsubsection{Deterministic result term}

% TODO

\subsubsection{Deterministic type of heap}

% TODO
% This proof was not really necessary, it seems...
% Look into this again

\subsubsection{Deterministic values of heap}

% TODO

\subsubsection{Uniqueness of results}

% TODO

\subsubsection{Normalization}

% TODO or omit

\subsection{Denotational semantics} % TODO review for latest progress

For the denotational semantics we started with the monadic
specification by Swierstra as a basis. The denotational semantics
would build on the monad, translating terms to GADT-defined monadic
values, which can in turn be evaluated.

First, we adapted the monad to work with a heap that is defined by
functions, instead of data types.

% TODO Is it?:
% Function representation of heap is not feasible, because the
% evaluation function needs to inspect it.

First attempt:

\begin{lstlisting}
     [[_]] : forall {ty f f'} -> (t : Term ty) -> St ty f f'
\end{lstlisting}

This lets you express the rule for allocation, but it is problematic
in other cases where a subexpression needs to be evaluated and
nothing is known about f'

% Second attempt:

\begin{lstlisting}
     [[_]] : forall {ty f} -> (t : Term ty) -> St ty f (extend t f)
\end{lstlisting}
where extend is an appropriate function

This lets the caller know the environment after evaluating the
subexpression, but it is not in constructor form and is therefore
hard to reason with.

\section{Hoare logic}

% TODO (Koen)

\section{Further work}\label{sec:further}

Besides having theorems about Hoare logic, the language could benefit from a `Hoare type system'. Having such a type system will enhance the ability of the programmer to specify properties about stateful programs.
% TODO refer to htt paper, bibtex?

The implemented language model is not sufficient for any non-trivial programming task. It lacks mechanisms for abstraction or non-trivial control flow. It could be extended with a control flow structure such as a while loop or with a fix point operator. It is desirable to preserve the strong normalization property, but this will put extra constraints on the language.

\end{document}

