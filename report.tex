\documentclass{article}
\usepackage{listings}
\usepackage{xcolor}
%\usepackage[utf8]{inputenc}
\lstset{language=haskell}
\lstset{moredelim=**[is][\color{blue}]{@}{@}}
\newcommand{\sectionref}[1]{section \ref{sec:#1}}
\author{Koen Wermer 3705951 \and Robert Hensing 3361063}
\title{Project report TPT}
\begin{document}

\maketitle

\tableofcontents

% TODO turn these into sections?? I guess this is fixed now.
%
%    Research question. Identify a clear and motivated question to answer.
%  somewhat obviously done in intro
%    Research contribution. Develop an implementation or new knowledge to answer the research question.
%  somewhat obviously done in intro
%    Research method. Discuss how you arrived at your answer.
%  somewhat obviously done in intro
%    Related work. Compare your answer with other contributions.
%  now has its own section after intro

\section{Introduction}

For this project, we have researched how a dependently typed language
like Agda can be used to model the semantics of a language with
mutable state using Agda as a proof checker.

The focus of our contribution is on the practical aspect of modeling,
to show and compare some possible approaches to modeling mutable state
in Agda.

We have extended a simplistic language with only boolean and number
types and no lambda abstractions or any other sort of control flow
structures.
An inductive definition of the language is provided in
\sectionref{def}.
In doing so, and proving theorems about this language, we have
encountered some challenges, which we will describe in
\sectionref{findings}.
Some possibilities we would have liked to explore, but were too
ambitious given the time frame, are described in \sectionref{further}.

\section{Related work}

\begin{description}
\item[Swierstra]\hfill

In his dissertation, Swierstra models some effects, of which, most prominently, mutable state. Agda is used to provide a total specification of effects, in order to be able to reason with it formally, which is a requirement for the implementation of Hoare Type Theory that is given.

\item[Pierce]\hfill

In this book includes an informal specification of mutable state.

\item[Nanevski et al.]\hfill

This paper introduces the concept of Hoare Type Theory, which allows tracking and enforcing of side effects. In the presence of polymorphism, it shows the soundness and compositionality of the theory, using separation logic to enable local reasoning.

\end{description}



\section{Language definition}\label{sec:def}

The term language we have extended with mutable state consists of
\begin{description}
\item[true, false, zero, succ] are normal forms for boolean and number constants
\item[if\_then\_else, iszero] are built-in functions
\end{description}

No abstraction or binding constructs are included.
We have extended the language with
\begin{description}
\item[var] is a normal form for representing variables
\item[ref] creates a new cell with an initial value
\item[:=] assigns a value to an existing cell
\item[!] reads a value from an existing cell
\item[$\mathbf{=>}$] make a pointer reference another cell
\item[\%] evaluate the first term, discarding its value and then evaluating the second term
\end{description}

\section{Findings}\label{sec:findings}

\subsection{Small step semantics}

\subsubsection{Deterministic result term}

We have proved that applying any two possible steps results in the
same term.  This proof, like the other two `deterministic' proofs
starts with case analysis on the possible steps. Although the type
checker can automatically omit and check bogus combinations based on
types, many cases remain.

\subsubsection{Deterministic type of heap}

We have proved that applying any two possible steps results in the same heap shape.
While this theorem seems useful for proving ``Deterministics values of heap'', there was actually no need to use it in that proof. It is a desirable property to have, so we have kept it.

\subsubsection{Deterministic values of heap}

We have proved that applying any two possible steps results in the
same heap state.  To prove this lemma, we had to make many attempts to
create type-correct pattern matches. In some cases we could not find a
suitable order of pattern matches that type-checked, so we had to
resort to case-specific lemma's.

\subsubsection{Uniqueness of results}

Using the `deterministic' lemma's, we were able to prove easily that terms result in unique tuples of normal forms and heap states.

\subsection{Denotational semantics} % TODO review for latest progress

For the denotational semantics we started with the monadic
specification by Swierstra as a basis. The denotational semantics
would build on the monad, translating terms to GADT-defined monadic
values, which can in turn be evaluated.

First, we adapted the monad to work with a heap that is defined by
functions, instead of data types.

First attempt:

\begin{lstlisting}
     [[_]] : forall {ty f f'} -> (t : Term ty) -> St ty f f'
\end{lstlisting}

This allows to express the rule for allocation, but it is problematic
in other cases where a subexpression needs to be evaluated and
nothing is known about f'

Second attempt:

\begin{lstlisting}
     [[_]] : forall {ty f} -> (t : Term ty) -> St ty f (extend t f)
\end{lstlisting}
where extend is an appropriate function

This lets the caller know the environment after evaluating the
subexpression, but it is not in constructor form and is therefore
somewhat hard to reason with. It did work better than the earlier
attempt, but we were not able to finish it.

In order to be able to implement the denotation semantics, we had to
extend the representation of values to be able to reference the heap
shape, because variable values have to carry the proof that they are
consistent with a heap.

\subsection{Hoare logic}

% TODO (Koen)

\section{Further work}\label{sec:further}

Besides having theorems about Hoare logic, the language could benefit from a `Hoare type system'. Having such a type system will enhance the ability of the programmer to specify properties about stateful programs. Although Swierstra did include an implementation of a Hoare Type System, we did not find a chance to adapt that to fit our model.

The implemented language model is not sufficient for any non-trivial programming task. It lacks mechanisms for abstraction or non-trivial control flow. It could be extended with a control flow structure such as a while loop or with a fix point operator. It is desirable to preserve the strong normalization property, but this will put extra constraints on the language.

\section{Literature}

\begin{itemize}
\item[] Swierstra, Wouter. A functional specification of effects. Diss. University of Nottingham, 2009.

\item[] Pierce, Benjamin C. Types and programming languages. MIT press, 2002.

\item[] Nanevski, Aleksandar, Greg Morrisett, and Lars Birkedal. Hoare type theory, polymorphism and separation. Journal of Functional Programming 18.5-6 (2008): 865-911.

\item[] Reynolds, John C. Separation Logic: A Logic for Shared Mutable Data Structures. LICS 2002.
\end{itemize}

\end{document}

